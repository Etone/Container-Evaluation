\chapter{Einleitung}
\label{chap:einleitung}

\section{Motivation}
\label{sec:motivation}
Die Welt wird immer stärker vernetzt. Durch den Drang, Anwendungen für viele Nutzer zugänglich zu machen besteht der Bedarf an Cloud-Diensten wie Amazon Web Services.
Eine dabei immer wieder auftretende Schwierigkeit ist es, die Skalierbarkeit der Services zu gewährleisten. Selbst wenn viele Nutzer gleichzeitig auf einen Service zugreifen, darf dieser nicht unter der Last zusammenbrechen.

Bis vor einigen Jahren wurde diese Skalierbarkeit durch \glspl{acr-vm} gewährleistet. Doch neben großem Konfigurationsaufwand haben \glspl{acr-vm} auch einen großen Footprint und sind für viele  Anwendungen zu ineffizient. Eine Lösung für dieses Problem stellen Container dar.

Diese Arbeit gibt einen Einblick in das Thema Container-Virtualisierung und beantwortet die Fragen, wie sich Docker als führende Technologie durchsetzen konnte, wie sich andere Technologien im Vergleich zu Docker schlagen und was die Zukunft in Form von Serverless-Technologien mit Bezug zu Containern bereithält.

\section{Aufbau der Arbeit}
\label{sec:aufbau}
Zu Beginn der Arbeit werden benötigte Grundlagen der Technologie erläutert. Dabei werden bestehende Container-Standards betrachtet und alle benötigten Kernel-Funktionen erklärt, die in Container-Runtimes Verwendung finden. Um einen besseren Einblick in die Technologie zu geben wird gezeigt, wie man mit Bash-Befehlen ohne Container-Runtime einen Prozess von einem Host-OS trennt. Dabei wird darauf eingegangen, wie eine eigene Dateihierarchie isoliert werden kann, wie Namespaces dabei helfen Funktionen des Linux-Kernels zu virtualisieren und wie der isolierte Prozess sicherer ausgeführt werden kann.

Im Anschluss wird die Frage beantwortet, wie Docker am verbreitetste Container-Engine wurde. Dazu wird die Geschichte der Technologie näher betrachtet.

\todo{Geschichte, wie wurde Docker zu Docker}
\todo{Vergleich Docker vs the World}
\todo{Aktuelle Probleme, Orchestrierung}
\todo{Serverless, wenn zeit reicht}