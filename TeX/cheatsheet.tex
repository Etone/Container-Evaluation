\chapter{Lookup}
\blindtext

Dieser Absatz wird zitiert von \citep{TheOpenContainerEssentialsVideoCollection}

Dieser Absatz ist ein bisschen Code
\lstset{language=java, caption=Deklarieren von Variablen in Java }
\begin{lstlisting}

	int i = 0;
	int a = 5;
	int c = a + i;
	//c = 5;
	System.out.println(c);
\end{lstlisting}

\blindtext

\fref{tab:Testtable} ist eine Tabelle mit booktabs und referenziert diese \\
\begin{table}[h]
	\begin{center}
		\begin{tabular}{lcr}
			\toprule
			Spalte 1 & Spalte 2 & Spalte 3 \\
			\midrule
			1 & 2 & 3 \\
			4 & 5 & 6 \\
			\bottomrule
		\end{tabular}
	\end{center}
	\caption{Testtabelle ohne Inhalt}
	\label{tab:Testtable}
\end{table}

Dieser Absatz verweist auf das Glossar
\gls{acr-k8} \emph{siehe} \gls{gls-k8}

\begin{table}[h]
	\begin{center}
		\begin{tabular}{|c|c|c|}
			\hline 
			\cellcolor{green} 1 & \cellcolor{red} 2 & \cellcolor{yellow} 3\\
			\hline 
			\cellcolor{red} 4 & \cellcolor{yellow} 5 & \cellcolor{green} 6\\
			\hline
		\end{tabular}
	\end{center}
\caption{Beispiel für cellcolor Matrizen}
\label{tab:comparisonMatrix}
\end{table}

Absatz Mathe: \fref{eq:simple} and \fref{eq:harder}

\begin{equation}
\label{eq:simple}
	\centering
	f(x) = (x+a)(x+b)	
\end{equation}

\begin{equation}
\label{eq:harder}
\centering
	L' = {L}{\sqrt{1-\frac{v^2}{c^2}}}
\end{equation}

\begin{equation}
	\centering
	J(\theta) = - \frac{1}{m} \Bigg[ \sum_{i=1}^{m}\sum_{k=1}^{K}y_k^i \log(h_{\theta}(x^i))_k + (1-y_k^i)\log(1-(h_{\theta}(x^i))_k) \Bigg] + \frac{\lambda}{2m}\sum_{l=1}^{L-1}\sum_{i=1}^{s_l}\sum_{j=1}^{s_l+1}(\theta_{ji}^l)^2
\end{equation}

\todo{Beispielhaftes Todo}

