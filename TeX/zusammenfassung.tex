
\chapter{Kurzfassung}
\label{chap:kurzfassung}

Diese Thesis behandelt einen Vergleich verschiedener Container-Technologien, wobei Container-Runtimes im Mittelpunkt stehen, um die Frage zu beantworten, warum Docker der aktuelle Branchenprimus ist. Dabei wurden verschiedene Container-Runtimes wie Docker, runc, rkt oder LXD genutzt, um eine eigene serviceorientierte Anwendung bereitzustellen. Docker und rkt bieten für diesen Fall viele Tools und sind deutlich spezifischer für diese Aufgabe ausgelegt sind. Runtimes wie LXC / LXD dienen vor allem dazu, Infrastruktur zu bieten. Im weiteren Verlauf wurden die Themen Orchestrierung, Sicherheit und die Serverless-Technologie betrachtet, wobei ein Ausblick auf aktuelle Anwendungsfälle der Container Technologie gegeben wird.

%entferne Seitenumbruch
\begingroup
\let\cleardoublepage\relax
\chapter{Abstract}
\label{chap:Abstract}

This thesis is about a comparison between different container technologies, where the focus is on container runtimes. For this, multiple runtimes like Docker, runc, rkt or LXD where used to deploy a serviceoriented application. Docker and rkt offer a lot of tools and are specifically designed for such tasks. Runtimes like LXC / LXD are designed to offer infrasturcture. Furthermore the topics orchestration, security and the serverless technology was looked at, where a look in the future for  use cases of the container technology is given.


Stichworte / Keywords: \textit{Container, Docker, OCI, runc, Kubernets, Serverless, Linux, rkt, LXD, gVisor}
\endgroup