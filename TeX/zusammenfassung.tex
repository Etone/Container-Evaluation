
\chapter{Kurzfassung}
\label{chap:kurzfassung}

Diese Thesis behandelt einen Vergleich verschiedener Container-Technologien, wobei Container-Runtimes im Mittelpunkt stehen. Dabei wurden verschiedene Container-Runtimes wie Docker, runc, rkt oder LXD genutzt, um eine eigene Serviceorientierte Anwendung bereitzustellen. Docker und rkt bieten für diesen Fall viele Tools und sind deutlich spezifischer für diese Aufgabe ausgelegt sind. Runtimes wie LXC / LXD dienen vor allem dazu, Infrastruktur zu bieten. Im weiteren Verlauf wurden die Dienste Kubernetes und CloudFoundry betrachtet, die es ermöglichen Container oder Software bereitzustellen. Auch wird sich mit der OCI, der CNCF und anderen Standards beschäftigt, die durch das aufsteigende Interesse an der Cloud unabdingbar geworden sind.

%entferne Seitenumbruch
\begingroup
\let\cleardoublepage\relax
\chapter{Abstract}
\label{chap:Abstract}

This thesis is about a comparison between different container technologies, where the focus is on container runtimes. For this, multiple runtimes like Docker, runc, rkt or LXD where used to deploy a serviceoriented application. Docker and rkt offer a lot of tools and are specifically designed for such tasks. Runtimes like LXC / LXD are designed to offer infrasturcture. In the further course a look was taken on the services Kubernetes and CloudFoundry, which provides the posibillity to deploy containers or software. Furthermore the topics OCI, CNCF and other standards will be looked into, since they became indispendsable with the rise of interest in the Cloud.

Stichworte / Keywords: \textit{Container, Docker, OCI, runc, Kubernets, Serverless, Linux, rkt, LXD, gVisor}
\endgroup